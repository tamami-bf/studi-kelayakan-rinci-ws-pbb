\documentclass[pdftex,12pt, oneside]{article}

%\usepackage[paperwidth=8.5in, paperheight=13in]{geometry} % Folio
\usepackage[paperwidth=8.27in, paperheight=11.69in]{geometry} % A4

\usepackage{makeidx}         % allows index generation
\usepackage{graphicx}        % standard LaTeX graphics tool
                             % when including figure files
\usepackage[bottom]{footmisc}% places footnotes at page bottom
\usepackage[english]{babel}
\usepackage{enumerate}
\usepackage{paralist}
\usepackage{float}
\usepackage{gensymb}  
\usepackage{listings}
\usepackage{mathtools} % atau \usepackage{amsmath}
\renewcommand{\baselinestretch}{1.5}

\newcommand{\HRule}{\rule{\linewidth}{0.5mm}}


\begin{document}
\sloppy % biar section ga melebar melewati kertas

\begin{center}
{\large STUDI KELAYAKAN RINCI PEMBANGUNAN SISTEM \textit{WEB SERVICES} SEBAGAI CARA KOMUNIKASI DENGAN TEMPAT PEMBAYARAN DALAM PENCATATAN PEMBAYARAN PAJAK BUMI DAN BANGUNAN PERDESAAN DAN PERKOTAAN DI KABUPATEN BREBES.}
\\[1cm]
29 Agustus 2016\\
Priyanto Tamami, S.Kom.
\end{center}

%\frontmatter%%%%%%%%%%%%%%%%%%%%%%%%%%%%%%%%%%%%%%%%%%%%%%%%%%%%%%


%%%%%%%%%%%%%%%%%%%%%%%%%%%%%%%%%%%%%%%%%%%%%%%%%%%%%%%%%%%%%%%%%%%%%%

\section{RUANG LINGKUP PEKERJAAN}




\begin{enumerate}[1.]
\item Wawancara


\item Pengumpulan Bahan-Bahan

\end{enumerate}

\section{SARANA DAN PRASARANA YANG MELIPUTI PERANGKAT KERAS DAN PERANGKAT LUNAK YANG DIPERLUKAN}


\section{SUMBER DAYA MANUSIA YANG TERLIBAT DALAM PENGOLAHAN DATA}


\section{ORGANISASI SISTEM PENGOLAHAN}


\section{WAKTU DAN BIAYA YANG DIBUTUHKAN DALAM PEMBUATAN/PENGEMBANGAN SISTEM PENGOLAHAN DATA SECARA MENYELURUH}


\section{MANFAAT DAN DAMPAK PENGOLAHAN DATA}



\end{document}