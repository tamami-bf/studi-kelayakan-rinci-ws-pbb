\documentclass[pdftex,12pt, oneside]{article}

%\usepackage[paperwidth=8.5in, paperheight=13in]{geometry} % Folio
\usepackage[paperwidth=8.27in, paperheight=11.69in]{geometry} % A4

\usepackage{makeidx}         % allows index generation
\usepackage{graphicx}        % standard LaTeX graphics tool
                             % when including figure files
\usepackage[bottom]{footmisc}% places footnotes at page bottom
\usepackage[english]{babel}
\usepackage{enumerate}
\usepackage{paralist}
\usepackage{float}
\usepackage{gensymb}  
\usepackage{listings}
\usepackage{mathtools} % atau \usepackage{amsmath}
\renewcommand{\baselinestretch}{1.5}

\newcommand{\HRule}{\rule{\linewidth}{0.5mm}}


\begin{document}
\sloppy % biar section ga melebar melewati kertas

\begin{center}
{\large STUDI KELAYAKAN RINCI PEMBANGUNAN SISTEM \textit{WEB SERVICES} SEBAGAI CARA KOMUNIKASI DENGAN TEMPAT PEMBAYARAN DALAM PENCATATAN PEMBAYARAN PAJAK BUMI DAN BANGUNAN PERDESAAN DAN PERKOTAAN DI KABUPATEN BREBES.}
\\[1cm]
29 Agustus 2016\\
Priyanto Tamami, S.Kom.
\end{center}

%\frontmatter%%%%%%%%%%%%%%%%%%%%%%%%%%%%%%%%%%%%%%%%%%%%%%%%%%%%%%


%%%%%%%%%%%%%%%%%%%%%%%%%%%%%%%%%%%%%%%%%%%%%%%%%%%%%%%%%%%%%%%%%%%%%%

\section{RUANG LINGKUP PEKERJAAN}

Ruang lingkup dari pekerjaan membangun sistem \textit{web services} sebagai bentuk atau cara komunikasi dengan tempat pembayaran dalam pencatatan pembayaran PBB-P2 di Kabupaten Brebes sebetulnya sudah cukup jelas, yaitu membangun sebuah layanan (atau dalam bahasa Teknologi Informasi adalah \textit{server}) yang menyediakan skema \textit{web services} namun bentuk yang lebih jelas dari penggunaan \textit{web services} disini akan berbentuk arsitektur \textit{Rest} dan hasil yang dikeluarkan oleh sistem aplikasi ini akan berformat JavaScript Object Notation (JSON).

Server nantinya hanya akan memberikan 3 (tiga) layanan standar yaitu : \textit{inquiry} untuk pengambilan informasi mengenai Objek Pajak maupun Subjek Pajak, pencatatan pembayaran, dan pembatalan pencatatan pembayaran.

Maksud dari ketiga layanan tersebut adalah sebagai berikut :

\begin{enumerate}
  \item \textit{Inquiry}
  
  Layanan ini digunakan apabila Bank sebagai tempat pembayaran ingin mengambil informasi mengenai objek pajak dan wajib pajak, untuk memastikan bahwa memang ada data dengan nomor objek tersebut, dapat pula digunakan sebagai bahan untuk verifikasi data, yaitu untuk memastikan apakah dengan nomor objek pajak tersebut, nama wajib pajak dan besarnya pajak bumi dan bangunan yang terhutang sesuai dengan yang tertera di dalam Surat Pemberitahuan Objek Pajak atau pernyataan dari Wajib Pajak / Kuasanya dalam membayar.
  
  \item Pencatatan Pembayaran
  
  Layanan ini digunakan oleh Bank sebagai tempat pembayaran untuk melakukan pencatatan pembayaran pada basis data yang berada di DPPK Kabupaten Brebes.
  
  \item Pembatalan Pembayaran
\end{enumerate}


\begin{enumerate}[1.]
\item Wawancara


\item Pengumpulan Bahan-Bahan

\end{enumerate}

\section{SARANA DAN PRASARANA YANG MELIPUTI PERANGKAT KERAS DAN PERANGKAT LUNAK YANG DIPERLUKAN}


\section{SUMBER DAYA MANUSIA YANG TERLIBAT DALAM PENGOLAHAN DATA}


\section{ORGANISASI SISTEM PENGOLAHAN}


\section{WAKTU DAN BIAYA YANG DIBUTUHKAN DALAM PEMBUATAN/PENGEMBANGAN SISTEM PENGOLAHAN DATA SECARA MENYELURUH}


\section{MANFAAT DAN DAMPAK PENGOLAHAN DATA}



\end{document}