\documentclass[pdftex,12pt, oneside]{article}

%\usepackage[paperwidth=8.5in, paperheight=13in]{geometry} % Folio
\usepackage[paperwidth=8.27in, paperheight=11.69in]{geometry} % A4

\usepackage{makeidx}         % allows index generation
\usepackage{graphicx}        % standard LaTeX graphics tool
                             % when including figure files
\usepackage[bottom]{footmisc}% places footnotes at page bottom
\usepackage[english]{babel}
\usepackage{enumerate}
\usepackage{paralist}
\usepackage{float}
\usepackage{gensymb}  
\usepackage{listings}
\usepackage{mathtools} % atau \usepackage{amsmath}
\renewcommand{\baselinestretch}{1.5}

\newcommand{\HRule}{\rule{\linewidth}{0.5mm}}


\begin{document}
\sloppy % biar section ga melebar melewati kertas

\begin{center}
{\large STUDI KELAYAKAN RINCI PEMBANGUNAN SISTEM \textit{WEB SERVICES} SEBAGAI CARA KOMUNIKASI DENGAN TEMPAT PEMBAYARAN DALAM PENCATATAN PEMBAYARAN PAJAK BUMI DAN BANGUNAN PERDESAAN DAN PERKOTAAN DI KABUPATEN BREBES.}
\\[1cm]
29 Agustus 2016\\
Priyanto Tamami, S.Kom.
\end{center}

%\frontmatter%%%%%%%%%%%%%%%%%%%%%%%%%%%%%%%%%%%%%%%%%%%%%%%%%%%%%%


%%%%%%%%%%%%%%%%%%%%%%%%%%%%%%%%%%%%%%%%%%%%%%%%%%%%%%%%%%%%%%%%%%%%%%

\section{RUANG LINGKUP PEKERJAAN}

Ruang lingkup dari pekerjaan membangun sistem \textit{web services} sebagai bentuk atau cara komunikasi dengan tempat pembayaran dalam pencatatan pembayaran PBB-P2 di Kabupaten Brebes sebetulnya sudah cukup jelas, yaitu membangun sebuah layanan (atau dalam bahasa Teknologi Informasi adalah \textit{server}) yang menyediakan skema \textit{web services} namun bentuk yang lebih jelas dari penggunaan \textit{web services} disini akan berbentuk arsitektur \textit{Rest} dan hasil yang dikeluarkan oleh sistem aplikasi ini akan berformat JavaScript Object Notation (JSON).

Server nantinya hanya akan memberikan 3 (tiga) layanan standar yaitu : \textit{inquiry} untuk pengambilan informasi mengenai Objek Pajak maupun Subjek Pajak, pencatatan pembayaran, dan pembatalan pencatatan pembayaran.

Maksud dari ketiga layanan tersebut adalah sebagai berikut :

\begin{enumerate}
  \item \textit{Inquiry}
  
  Layanan ini digunakan apabila Bank sebagai tempat pembayaran ingin mengambil informasi mengenai objek pajak dan wajib pajak, untuk memastikan bahwa memang ada data dengan nomor objek tersebut, dapat pula digunakan sebagai bahan untuk verifikasi data, yaitu untuk memastikan apakah dengan nomor objek pajak tersebut, nama wajib pajak dan besarnya pajak bumi dan bangunan yang terhutang sesuai dengan yang tertera di dalam Surat Pemberitahuan Objek Pajak atau pernyataan dari Wajib Pajak / Kuasanya dalam membayar.
  
  \item Pencatatan Pembayaran
  
  Layanan ini digunakan oleh Bank sebagai tempat pembayaran untuk melakukan pencatatan pembayaran pada basis data yang berada di DPPK Kabupaten Brebes. Layanan ini akan otomatis melakukan pencatatan pembayaran pada basis data SISMIOP dengan tentu saja memberikan respon berupa informasi nomor objek mana yang dicatatkan pembayarannya, atas nama siapa pajak bumi dan bangunan terbayarkan, dan berapa jumlah tagihan yang terbayar.
  
  \item Pembatalan Pembayaran
  
  Layanan ini digunakan oleh Bank sebagai tempat pembayaran untuk melakukan pembatalan pembayaran apabila proses pencatatan pembayaran yang diminta oleh Bank sebagai tempat pembayaran telah mencapai batas waktu \textit{timeout} yang telah ditetapkan oleh Bank tetapi layanan \textit{web services} yang berada di DPPK belum memberikan respon. Atau ada kondisi lain yang menyebabkan pencatatan pembayaran harus dibatalkan.
  
\end{enumerate}

Hal-hal yang mendukung studi kelayakan dibangunnya sistem \textit{web services} untuk pencatatan pembayaran PBB-P2 adalah dari kondisi-kondisi berikut :

\begin{enumerate}[1.]
\item Wawancara

Hal ini dilakukan guna memperkuat kebutuhan akan adanya layanan pencatatan pembayaran yang akurat dan dengan data terkini (atau dengan istilah lain adalah \textit{realtime}). Ini dibutuhkan terutama untuk pengambilan keputusan atau kebijakan bila misalkan suatu saat dilakukan kegiatan intensifikasi ke sebuah Desa / Kecamatan, akan terlihat efektif bila dalam masa beberapa waktu setelah kegiatan intensifikasi, seharusnya ada peningkatan realisasi pembayaran untuk Desa / Kecamatan tersebut.

Dari hasil wawancara maka dapat disimpulkan bahwa beberapa keinginan yang diharapkan dari sistem pencatatan \textit{realtime} ini (atau dikenal dengan sistem \textit{host-to-host}), adalah sebagai berikut :

\begin{enumerate}
  \item Tersedianya sumber data yang akurat dan konsisten untuk menunjukkan piutang dan realisasi.
  
  \item Dapat tersedianya realisasi pembayaran waktu nyata atau \textit{realtime} guna pengambilan keputusan / kebijakan.
  
  \item Pembayaran yang tercatat langsung secara waktu nyata atau \textit{realtime} dapat digunakan sebagai informasi pertanggungjawaban langsung kepada wajib pajak atau kuasanya dalam melakukan kontrol terhadap pembayaran-pembayaran yang dilakukan melalui petugas pemungut.
\end{enumerate}

\item Pengumpulan Bukti Pendukung

Data pendukung lain untuk memperkuat asumsi-asumsi atau pendapat dari hasil wawancara adalah dengan pengumpulan bukti-bukti yang dapat disimpulkan sebagai berikut :

\begin{enumerate}

\item Bahwa data antara SISMIOP (basis data yang berada di DPPK), dengan Cash Management System (CMS) yaitu basis data yang berada di Bank sebagai tempat pembayaran, ada selisih baik kondisi pada jumlah objek piutang dan besarnya nilai piutang, serta jumlah objek pembayaran dan besarnya nilai pembayaran yang tercatat.

\item Bahwa masih adanya penyalahgunaan kewenangan petugas pemungut dalam melakukan pemungutan PBB-P2 dan penyetoran hasil pemungutannya ke Bank sebagai tempat pembayaran.

\item Bahwa masih adanya selisih waktu pencatatan pembayaran antara basis data yang berada di DPPK dan Bank sebagai tempat pembayaran, yang ini tidak dapat diubah karena menjadi keterbatasan sistem bahwa data-data nomor objek pajak yang telah dibayarkan baru dapat diambil dan dipindahkan ke sistem basis data SISMIOP di DPPK setelah melalui waktu \textit{cut-off} yang ditetapkan Bank sebagai tempat pembayaran.

\end{enumerate}

\end{enumerate}

Dari data-data tersebut, maka sudah seharusnya bentuk sistem \textit{host-to-host} dibangun, karena bukan hanya akan mendapatkan nilai realisasi pembayaran dalam waktu nyata (\textit{realtime}) melainkan konsistensi data yang terjaga dalam satu basis data. Sebagai nilai lebih adalah dapat digunakan sebagai sebuah alat atau media pertanggungjawaban petugas pemungut PBB-P2 dalam melakukan penyetoran hasil pemungutannya ke Bank sebagai tempat pembayaran.

\section{SARANA DAN PRASARANA YANG MELIPUTI PERANGKAT KERAS DAN PERANGKAT LUNAK YANG DIPERLUKAN}

Sebagai sebuah sistem yang utuh, maka diperlukan subsistem pembentuknya agar keseluruhan sistem dapat bekerja dengan baik. Tentu karena yang dibangun adalah sebuah sistem layanan (\textit{service}) yang menggunakan teknologi informasi, maka tidak akan lepas dari kebutuhan perangkat lunak dan perangkat keras yang mendukungnya. Kebutuhan akan perangkat lunak dan perangkat keras tersebut disebutkan sebagai berikut :

\begin{enumerate}

\item \textbf{Perangkat Lunak}

Perangkat lunak sebagai pembentuk sistem layanan \textit{web services} ini berjalan adalah sebagai berikut :

\begin{enumerate}
  \item Basis Data
  
  Sebagaimana sebuah sistem yang melakukan pengolahan data, maka diperlukan tempat untuk menyimpan data-data tersebut yang biasa dikenal dengan basis data (\textit{database}). Karena sistem layanan \textit{web services} ini akan mencatatkan langsung permintaan sebuah transaksi pembayaran, maka basis data yang digunakan oleh sistem layanan \textit{web services} ini sama dengan basis data yang digunakan dalam pengelolaan objek dan subjek pajak PBB-P2, yaitu SISMIOP, sistem basis data ini menggunakan sistem basis data Oracle 11g yang tentu saja berada di DPPK Kabupaten Brebes. Hanya saja, dalam pengoperasiannya, perlu adanya perubahan struktur basis data di SISMIOP atau lebih tepatnya penambahan tabel dan prosedur dalam basis data SISMIOP agar layanan \textit{web services} dapat mencatatkan transaksi pembayaran lebih cepat dan historis dari kegiatan layanan \textit{web services} ini tercatat.
  
  \item \textit{Text Editor}
  
  Tentunya untuk membangun sebuah sistem aplikasi yang terdiri dari susunan kode-kode program, dibutuhkan \textit{text editor} atau \textit{Integrated Development Environment} (\textit{IDE}), untuk penggunaan \textit{text editor} sendiri banyak yang menawarkan secara gratis, mulai dari vi, vim, emacs, notepad, atau perangkat lunak lain yang bertugas sebagai \textit{text editor} pasti dapat digunakan sebagai salah satu alat untuk membangun sistem aplikasi ini. Untuk IDE adalah sebuah perangkat lunak dengan dukungan fasilitas yang lebih lengkap dibandingkan sebuah \textit{text editor} biasa, hal umum yang biasa ditawarkan oleh sebuah IDE adalah \textit{highligth} dari kode-kode yang ditulis sehingga memudahkan dalam pembacaan kode program, bahkan ada yang menyertakan fasilitas \textit{code snippets} sehingga seorang pemrogram tidak perlu terlalu sering untuk melihat dokumentasi sebuah modul karena secara otomatis, IDE akan memberikan informasinya seketika saat pemrogram kebingungan tentang apa nama fungsi atau prosedur  yang akan digunakan.
  
  IDE ini pun banyak yang menyediakan secara gratis dan selalu diperbaharui secara berkala. Salah satunya adalah Eclipse.
  
  \item Pustaka (\textit{Library})
  
  Pustaka atau \textit{library} adalah modul modul pendukung terbangunnya sebuah sistem aplikasi layanan \textit{web services} ini. Pustaka atau \textit{library} yang dibutuhkan sistem aplikasi ini cukup menggunakan yang gratis, namun tentu saja, handal dalam melaksanakan tugasnya.
  
  Sebagai inti yang membangun sistem aplikasi ini, nantinya akan menggunakan \textit{framework} Spring, Spring ini merupakan \textit{framework} yang digunakan untuk membangun sebuah aplikasi \textit{enterprise}, termasuk \textit{framework} yang ringan untuk mendukung secara penuh dalam pengembangan aplikasi \textit{enterprise}.
  
  Spring dapat digunakan untuk melakukan pengaturan deklarasi manajemen transaksi, \textit{remote access} dengan menggunakan RMI, atau layanan web lainnya. Spring juga memungkinkan untuk menggunakan hanya modul-modul tertentu sehingga tidak perlu menggunakan semua modul spring dalam aplikasi apabila tidak diperlukan.
  
  Adapun fitur yang ditawarkan oleh \textit{framework} spring adalah sebagai berikut :
  
  \begin{itemize}
    \item \textit{Transaction Management}, dimana Spring \textit{framework} menyediakan sebuah lapisan / \textit{layer} abstrak yang umum untuk manajemen transaksi, sehingga memudahkan \textit{developer} dalam melakukan manajemen transaksi.
    \item JDBC \textit{Exception Handling} : yaitu lapisan / \textit{layer} abstrak JDBC yang menawarkan \textit{exception} yang bersifat hierarki sehingga memudahkan penanganan kesalahan.
    \item Spring menawarkan layanan integrasi yang terbaik dengan Hibernate, JDO, dan iBatis.
    \item MVC \textit{Framework} : dimana Spring hadir dengan \textit{framework} aplikasi web MVC yang dibangun di atas inti Spring. Spring merupakan \textit{framework} yang sangat fleksibel dalam pengaturan strategi \textit{interface}, dan mengakomodasi beberapa teknologi \textit{view} seperti JSP, Velocity, Tiles, iText, dan POI.
  \end{itemize}
  
  Arsitektur dari \textit{framework} Spring ini terbagi menjadi beberapa bagian, yaitu sebagai berikut :
  
  \begin{enumerate}[1)]
    \item Spring Core Container
    
    Spring \textit{Core Container} terdiri dari modul-modul \textit{spring-core}, \textit{spring-beans}, \textit{spring-context}, \textit{spring-context-support}, dan \textit{spring-expression}.
    
    Modul \textit{spring-core} dan \textit{spring-beans} adalah dasar dari \textit{framework} Spring ini, termasuk di dalamnya adalah fitur IoC dan \textit{Dependency Injection}. \textit{BeanFactory} adalah implementasi paling mutakhir dari pola bawaanya. Kondisi ini menghapuskan kebutuhan pembuatan kerangka program dan menjadikan pemisahan antara ketergantungan konfigurasi dan spesifikasi dari logika program sesungguhnya.
    
    Modul \textit{spring-context} digunakan untuk mengakses objek dengan cara \textit{framework} yang mirip dengan cara JNDI \textit{registry}. Sedangkan modul \textit{spring-context-support} memberikan dukungan integrasi pada pustaka / \textit{library} pihak ke-3 untuk berjalan pada aplikasi Spring untuk \textit{caching} (seperti menggunakan EhCache, Guava, JCache), \textit{mailing} (seperti JavaMail), \textit{scheduling} (seperti CommonJ dan Quartz), dan \textit{template engines} (seperti FreeMarker, JasperReports, Velocity).
    
    Sedangkan modul \textit{spring-expressiong} memberikan kemampuan handal dalam bahasa Expression untuk melakukan \textit{query} dan manipulasi objek grafik saat \textit{runtime}.
    
    \item Spring AOP dan Instrumentasi
    
    Modul \textit{spring-aop} memberikan kemampuan untuk melakukan AOP (Aspect Oriented Programming) yang implementasinya memungkinkan untuk mendefinisikan \textit{method interceptor} dan \textit{pointcut} untuk memperjelas pemisahan kode yang mengimplementasikan fungsionalitas yang seharusnya dipisahkan.
    
    Secara terpisah modul \textit{spring-aspects} memberikan integrasi terhadap AspectJ.
    
    Sedangkan modul \textit{spring-instrument} memberikan dukungan instrumen yang dapat digunakan dibeberapa aplikasi \textit{server}.
    

    \item Spring Messaging    
    
    Sejak \textit{Framework} Spring 4, terdapat modul \textit{spring-messaging} dengan abstraksi kunci dari \textit{Spring Integration} seperti \textit{Message}, \textit{MessageChannel}, \textit{MessageHandler}, dan lainnya untuk memberikan layanan bagi aplikasi yang berbasis pesan (\textit{messaging}). Modul ini juga menyertakan beberapa anotasi untuk memetakan pesan ke \textit{method}, mirip seperti model pemrograman anotasi di Spring MVC.
    
    
    \item Spring Data Access / Integration
    
    baca disini -
    \begin{verbatim}
    http://docs.spring.io/spring/docs/current/spring-framework-reference/html/overview.html#overview-core-container
    \end{verbatim}
    
    \item Spring Web
    
    
    \item Spring Test
    
    
  \end{enumerate}
  
  \item \textit{Servlet Container}
\end{enumerate}

\item \textbf{Perangkat Keras}

Perangkat keras yang membentuk sistem layanan \textit{web services} ini berjalan adalah sebagai berikut :

\begin{enumerate}
  \item Server Basis Data
  \item Server Aplikasi
  \item Router
  \item VPN Server
\end{enumerate}

\item \textbf{Akses Internet}

Hal lain yang mendukung sistem layanan ini dapat bekerja adalah akses internet dengan kemampuan yang mencukupi. Artinya, lebar pita \textit{bandwidth} harus mencukupi untuk transaksi menerima dan memberikan respon.

Sebagai contoh simulasi kasar dalam penggunaan \textit{bandwidth} akses internet ini adalah sebagai berikut, dari sejarah data yang terekam sejak pengalihan PBB-P2 ke Daerah, transaksi terbanyak dalam periode harian adalah 21.165 transaksi, yaitu ditanggal 29 Agustus 2014, bila dalam sekali proses pencatatan transaksi pembayaran PBB-P2, data yang dikirim oleh Bank sebagai tempat pembayaran hanya berupa Nomor Objek Pajak dan Tahun Pajak yang diinginkan, Nomor Objek Pajak berjumlah 18 karakter, dan tahun pajak berjumlah 4 karakter, dengan total 22 karakter, jika 1 karakter = 1 byte, maka dibutuhkan 22 byte untuk sekali mengunduh atau menerima data transaksi, untuk tanggal kritis seperti di tanggal 29 Agustus 2014 kebutuhan akan unduh atau menerima data permintaan pencatatan pembayaran data transaksi hanya sebesar 455 kB (\textit{kilo bytes}). Namun berbeda untuk kasus respon atau mengunggah data hasil pencatatan transaksi pembayaran yang sukses, bila data yang dikirim dari server DPPK ke Server Bank sebagai tempat pembayaran memiliki struktur berikut :

\begin{itemize}
  \item Nomor Objek Pajak = 18 karakter
  \item Tahun Pajak = 4 karakter
  \item Nomor Transaksi Penerimaan Daerah = 16 karakter
  \item Jumlah Pembayaran Pokok = 12 karakter
  \item Jumlah Pembayaran Denda = 12 karakter
  \item Nama Wajib Pajak = 30 karakter
  \item Alamat Objek Pajak = 45 karakter
\end{itemize}

Maka untuk menjawab atau merespon sebuah permintaan transaksi pembayaran akan membutuhkan lebar pita sebanyak 137 karakter atau 137 \textit{byte}. Dengan perhitungan terburuk, dalam rangka menjaga ketersediaan layanan dalam merespon suatu permintaan, maka bahwa transaksi pada tanggal 29 Agustus 2014 terjadi pada waktu yang bersamaan, akan dibutuhkan lebar pita (\textit{bandwidth}) sebanyak 2.899.605 \textit{byte}, bila 1 kB adalah 1.024\textit{byte}, dan 1 MB adalah 1.024 \textit{kB}, maka dihasilkan kebutuhan lebar pita (\textit{bandwidth}) sebesar 2,77 MB. Ini adalah lebar pita (\textit{bandwidth}) minimal yang dibutuhkan, untuk menjaga apabila suatu saat kedepan akan ditemukan jumlah transaksi yang jauh lebih besar dari kejadian di tanggal 29 Agustus 2014.

Lebar pita atau \textit{bandwidth} yang diperlukan ini adalah lebar pita atau \textit{bandwidth} murni, karena di beberapa penyedia layanan, ada beberapa paket yang lebar pitanya atau \textit{bandwidth} diatur dalam model berbagi (\textit{sharing bandwidth}), yang biasanya ada pada paket internet untuk rumah tangga, dimana \textit{bandwidth} atau lebar pita yang semisal ditawarkan 10 MB/s, maka pada kenyataan kecil kemungkinan mencapai 10 MB/s, karena \textit{bandwidth} yang ditetapkan oleh penyedia layanan, misalkan 100 MB/s akan dibagi kesesama pengguna paket 10 MB/s, yang pada akhirnya, apabila ada lebih dari 10 pelanggan menggunakan paket 10 MB/s atau kita tetapkan secara kasar bahwa ada 100 pelanggan menggunakan paket 10 MB/s, maka apabila dalam waktu bersamaan setiap pengguna menggunakan internet, rata-rata setiap pelanggan hanya akan mendapatkan \textit{bandwidth} atau lebar pita sebesar 100 pelanggan dibagi 100 MB/s, atau hanya sebesar 1 MB/s saja. Ini berbeda dengan layanan yang menyediakan \textit{bandwidth} secara independen untuk tiap pelanggan, atau dalam istilah yang kita kenal adalah paket internet \textit{dedicated} yang seharusnya, \textit{bandwidth} atau lebar pita yang diberikan murni sesuai jumlah pada saat ditawarkan.

Selain lebar pita (\textit{bandwidth}) seperti dijelaskan diatas, untuk menjaga ketersediaan sambungan / koneksi, maka diperlukan 2 (dua) atau lebih sambungan atau koneksi internet dengan penyedia jasa yang berbeda. Hal ini untuk menjaga apabila salah satu koneksi internet mengalami kegagalan sambungan atau ada permasalahan lain sehingga koneksi gagal terjalin, maka sambungan atau koneksi dari penyedia jasa yang lain yang masih memiliki koneksi akan menggantikan posisi jaringan yang sedang bermasalah. Sehingga dalam fungsinya sebagai penyedia layanan \textit{web services} dapat memberikan layanan yang maksimal selama 24 jam sehari, sepanjang 7 hari dalam seminggu.

Skenario lain yang dapat digunakan untuk menjaga ketersediaan sambungan atau koneksi atau dalam istilah kerennya adalah \textit{availability} yaitu dengan membangun \textit{server} lain yang berbeda lokasi secara geografis. Tentunya membangun sebuah \textit{server} di luar DPPK Kabupaten Brebes akan membutuhkan biaya yang tidak sedikit. Namun dengan perkembangan teknologi informasi sekarang ini, layanan yang mungkin digunakan secara lebih efektif dan efisien adalah menggunakan layanan \textit{cloud} atau komputer awan. Layanan \textit{cloud} ini sebetulnya berbentuk \textit{server} virtual yang secara fisik terdiri dari beberapa server yang digabung menjadi sebuah layanan \textit{cloud}, sebuah \textit{server} virtual di \textit{cloud} akan memanfaatkan \textit{resources} atau sumber daya yang optimal dari \textit{server-server} fisik yang menyusunnya.

Dengan menggunakan layanan \textit{cloud} ini, maka bukan hanya melakukan penjagaan terhadap koneksi internet, tetapi juga melakukan duplikasi data sehingga apabila ada kerusakan data di \textit{server} basis data DPPK maka data tetap utuh berada di server basis data \textit{cloud}. Secara kasar mungkin akan terlihat mirip dengan sistem \textit{single host} yang sekarang digunakan, yaitu ada \textit{server} basis data di DPPK, dan ada \textit{server} basis data di Bank tempat pembayaran, yang membedakan adalah pada sistem \textit{single host}, struktur basis data di DPPK dan struktur basis data di Bank tempat pembayaran berbeda sangat jauh, sedangkan pada layanan \textit{cloud} antara struktur basis data di DPPK dan struktur basis data di \textit{server cloud} sama persis sehingga dapat dilakukan teknik \textit{mirroring} atau penyamaan data antara basis data di DPPK dan layanan \textit{server cloud}.

\end{enumerate}


\section{SUMBER DAYA MANUSIA YANG TERLIBAT DALAM PENGOLAHAN DATA}


\section{ORGANISASI SISTEM PENGOLAHAN}


\section{WAKTU DAN BIAYA YANG DIBUTUHKAN DALAM PEMBUATAN/PENGEMBANGAN SISTEM PENGOLAHAN DATA SECARA MENYELURUH}


\section{MANFAAT DAN DAMPAK PENGOLAHAN DATA}



\end{document}